\section{Data mining}
\label{sec:datamining}
Why did data mining was developed?
This chapter can give some context about that and then introduces some building blocks of the data mining framework.
\subsection{The problems}
\label{sub:the_problems}
The invention of the computer changed the way we think about storing and managing data.
Unlike books in the library or merchants in the store, data in computers can grow exponentially and instantaneously at a rate like never before.
The need for a systematic way to collect and extract useful information from a big data set started to coin in the late 1980s within companies' research departments \citep{coenen_datamining_2011}.

One of the first problems that data mining tried to solve is to create decision supports from retail clients' transactions and the sales information \citep{coenen_datamining_2011}.
The aim is to drive the sales up, by giving out suggestions, promotions, and special pricing to the targeted customer based on their behaviors.
For example, retailers can use the system to find out which items are frequently bought together, then arrange them close to each other to create a reminding effect, which can increase the sale.
Advance a few decades later, Netflix - a streaming service company - created a system to recommend movies to users based on their favorites and activities \citep{netflix_rs_2016}.
The most difficult part of such a system is that the data is often significantly smaller than the search space.
An average person can only watch a limited number of movies, while the total number of movies is vastly bigger.
Along with that, nowadays, with the raising of low-cost communicable devices and sensors, we need to find efficient ways to deal with the data produced by them \citep{data_mining_iot_2014}.
Hence, more sophisticated and clever methods are needed; and many have been invented to deal with the growing of the complexities of the problems.

These are just a few examples of some problems that emerged in the modern time of computing and data mining.
As we can imagine, the potential of data mining is unbounded.

\subsection{The main building blocks}
\label{sub:building_blocks}

There are three main phases of data mining: \textit{data collection}, \textit{data wrangling}, and \textit{analytical processing}.

Data collection usually involves the use of hardware or software to acquire raw data.
This could be sensors' data of the environment, or user activities data from computer applications.
The choice of which data to collect is crucial and can affect greatly the quality of the result in the later phases.

After the data is collected, it is usually in a form that is difficult to be processed directly by the algorithms.
That's why we need the data preprocessing phase to make the data easier to be consumed.
This could be structuring the data into known format, e.g. multidimensional formats, time series, etc.; or removing corrupted data.

Analytical processing is the most interesting phase where useful information or insights start to emerge.
From the analytical perspective, we can categorize data mining into four "super problems": clustering, classification, \ac{apm} and outlier analysis \citep{Aggarwal15}.
Even though mining processes are different from each other, they often share some similarities and common patterns that we can generalize and apply similar techniques to them.
For example, association patterns are somewhat similar to overlapping clusters, where each pattern is corresponding to a cluster.

In the scope of this thesis, we will be more interested on the \acl{apm} problem.
\Acl{fim} \citep{borgelt_fim_2012} is the most popular model of \acl{apm}.

% Maybe introduce more type of problems here (Aggarwal15 4.1)