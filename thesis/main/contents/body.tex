

\chapter{Introduction and problem statement}
\label{cha:intro}
With the enormous amount of data we produce nowadays, data mining is becoming more and more prevalent.
Consequently, the goal with modern data mining methods is not only to discover information from crude data, but also to condense that information down to concise descriptions and insights.
One such method is redescription mining~\citep{ramakrishnan_turning_2004}.
In a nutshell, it aims to find different ways to describe the same things.

\chapter{Background Knowledge}
\label{cha:background}
In this chapter we will explore the underlying theory of the thesis.
We will go from more general knowledge to more specific concepts.
Firstly, the data mining idea and framework are introduced.
Then we will dig deeper into frequent itemset mining and redescription mining, which are the two main concepts that the thesis is about.
\section{Data mining}
Why did data mining come up?
We will address that question and then introduce some building blocks of the data mining framework in this chapter.
\subsection{The problems}
\label{sub:the_problems}
The invention of the computer changed the way we think about storing and managing data.
Unlike books in the library nor merchants in the store, data in computer can grow exponentially and instantaneously like never before.
The need of a systematic way to collect and extract useful information from a big data set started to coin in the late 1980s within company research department  \citep{coenen_datamining_2011}.

One of the first problems that data mining tried to solve is to create decision supports from retail clients transactions and the sales information \citep{coenen_datamining_2011}. 
The aim is to drive the sales up, by giving out suggestions, promotions and special pricing to the targeted customer based on their behaviors.
For example, retailers can use the system to find out which items are frequently bought together, then arrange them close to each other to create a reminding effect, which can increase the sale.
Advance a few decades later, Netflix - a streaming service company - created a system to recommend movies to users based on their favorites and activities \citep{netflix_rs_2016}.
The most difficult part of such system is that the data is often significantly smaller than the search space.
An average person can only watch a limited number of movies, while the total number of movies are vastly bigger.
Nowadays, with the rising of low-cost communicable devices and/or sensors, we need to find efficient ways to deal with the data produced by them \citep{data_mining_iot_2014}.
% TODO improve this

These are just a few examples of some problems that emerged in the modern time of computing and data mining.
As we can imagine, the potential of data mining is unbounded.

\subsection{The main building blocks}
\label{sub:building_blocks}
There are three main phases of data mining: \textit{data collection}, \textit{data preprocessing}, and \textit{analytical processing}.

Data collection usually involving the use of hardware or software to collect raw data.
This could be sensors' data of the environment, or user activities data from the application.
The choice of which data to collect is crucial and can affect greatly of the quality of the result in the later phases.

After the data is collected, it's usually in a form that's difficult to consume directly by the algorithms.
That's why we need the data preprocessing phase to make the data to be easier to be consumed.
This could be structuring the data into known format, e.g. multidimensional format, time series, etc.; or removing bad data.

Analytical processing is the most interesting phase where the useful information or insights start to emerge.
Even though mining processes are different from each other, they often share some similarities and common patterns that we can generalize and apply similar techniques to them.
One of the technique that we will be focusing on in this thesis is the \textit{frequent itemset mining} \citep{borgelt_fim_2012}.

\section{Frequent itemset mining}
\acrfull{fim} was originally developed for marketing purpose, as introduced in \autoref{sub:the_problems}.
The purpose of the algorithm back then was to try to analyze the sale's transactions data to extract information about which items are frequently bought together.
Nowadays the applications of 
\subsection{Definitions}

% \begin{definition}[Test]
%     Something here
% \end{definition}
\subsection{Apriori}
\subsection{ECLAT}

\section{Redescription mining}


\chapter{Employing ECLAT for Redescription Mining}
\label{cha:employment}

\chapter{Experiments}
\label{cha:experiments}

\chapter{Conclusions}
\label{cha:conclusions}