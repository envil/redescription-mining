 \documentclass[12pt,a4paper,draft]{article}
\usepackage[utf8]{inputenc}
\usepackage{amsmath}
\usepackage{amsfonts}
\usepackage{amssymb}
\usepackage{gensymb}
\author{Viet Ta}
\title{Redescription Mining from Boolean data with a hierarchy}
\begin{document}
\maketitle

With the enormous amount of data we produce nowadays, data mining is becoming more and more prevalence. Consequently, the modern idea of data mining is not only to discover information from crude data, but also to condense that information down to concise descriptions and insights. One method of such "insights mining" is the redescription mining.

The intuition of redescription mining is to find two different ways to describe the same thing. Before we can go for the definition of it, several fundamental terminologies need to be addressed.
Firstly, the redescription mining data model is a tuple of 3 sets: $entities,\ attributes,$ and $views$. Each entity is associated with a set of $attributes$. The $attributes$ are partitioned into disjoint $views$. Intuitively, this model is analogous to the concept of data table, where $entities$ are equivalent to rows, $attributes$ correspond to columns, one view is just set of column(s). Secondly, a description is simply a query evaluates each entity and output a Boolean value. And thirdly, a support of a query is a set of entities in the data that renders true in the query.
In a nutshell, a redescription is a pair of descriptions with disjoint views and sufficiently similar supports. % cite here
The similarity of two supports can be measured by a distance function. If the similarity is above a threshold then the two descriptions form a redescription.

One classic example of redescription mining is about finding bio-climatic niche. In the example, we want to find out the relation between temperature and the distribution of the lynxes on the map. The first query is to find the geographical area where the maximum March temperature is in a specific range. The second query is to find the geographical area on the map where the lynxes inhabit. If we can find a pair of queries, one from each set of queries, such that the results of them, i.e spatial data, overlap at a certain predefined threshold; then we have successfully found a redescription.

% to here

If we keep going deeper from here, redescription mining can be branched out by the way we define the data structure, special constrains, etc. Among various types of data those can be mined, Boolean data is a common one, which requires it own techniques and tricks to deal with.
%Boolean data structure
The problem of Boolean data with a hierarchy is an intriguing scenario where an additional dimension of data structure is introduced.
\end{document}