 \documentclass[12pt,a4paper,draft]{article}
\usepackage[utf8]{inputenc}
\usepackage{amsmath}
\usepackage{amsfonts}
\usepackage{amssymb}
\usepackage{gensymb}
\author{Viet Ta}
\title{Redescription Mining from Boolean data with a hierarchy}
\begin{document}
\maketitle

In the contemporary world of unprecedented amount of data we produce every second, data mining is an essential practice for the new wave of technological innovations to happen. Consequently, the modern idea of data mining is not only to discover information from crude data, but also to condense that information down to concise descriptions and insights. One solid representation of such "insights mining" is the redescription mining.

The intuition of redescription mining is to find two different ways to describe the same thing.
In a nutshell, a redescription is a pair of descriptions, i.e. queries, with disjoint views and sufficiently similar supports. % cite here
% SAY ABOUT SUPPORT (you need to explain a bit what "support" means and why this might be valuable and of interest to the analyst).
A support of a query is a set of entities in the data that renders true in the query. The similarity of two supports can be measured by a distance function, namely Jaccard distance. Along with that, each description has a view which is a set of attribute of the data; we have a constrain of which the views must not intersect each other.
%Although, to fully understand the concept, a framework of theories is prerequisite.

One classic example of redescription mining is about finding bioclimatic niche. In the example, we want to find out the relation between temperature and the distribution of the lynxes on the map. The first query is to find the geographical area where the maximum March temperature is in a specific range, e.g. \(-24.4 \degree C \leq T \leq 3.4\degree C\). The second query is to find the geographical area on the map where the lynxes inhabit. If we can find a pair of queries, one from each set of queries, such that the results of them, in this case spatial data, overlap at a certain predefined threshold; then we have successfully found a redescription.

If we keep going deeper from here, redescription mining can be branched out by the way we define structure the data, special constrains, etc. Among various types of data those can be mined, Boolean data is a common one, which requires it own techniques and tricks to deal with.

\end{document}